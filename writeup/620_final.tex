
\documentclass[11pt]{article}


% Use wide margins, but not quite so wide as fullpage.sty
\marginparwidth 0.5in 
\oddsidemargin 0.25in 
\evensidemargin 0.25in 
\marginparsep 0.25in
\topmargin 0.25in 
\textwidth 6in \textheight 8 in
% That's about enough definitions

% multirow allows you to combine rows in columns
\usepackage{multirow}
% tabularx allows manual tweaking of column width
\usepackage{tabularx}
% longtable does better format for tables that span pages
\usepackage{longtable}
\usepackage{amsmath}

\begin{document}

\author{Alex Burka, Sarah Costrell, Conor O'Brien}
\title{MEAM 620 Project 3A}
\maketitle

\section{Description of Problem and Associated Algorithms}

\subsection{CAPT}
The concurrent assignment and planning of trajectories problem, or CAPT, involves finding a method of assigning $N$ homogeneous robots to $M$ goals and generating collision-free paths in order to reach the goals. The linear assignment portion of this problem may be offloaded to the Hungarian Algorithm, which is of complexity order $\mathcal{O}(N^3)$.

\subsection{C-CAPT}
C-CAPT is a centralized solution to the CAPT problem, via which trajectories are minimized via a cost functional encompassing valid assignment, resource utilization (with respect to the assignment matrix), initial conditions, terminal conditions, robot capabilities (the dynamics of each robot, generally assumed to be first-order), and collision avoidance.

\subsection{D-CAPT}




\section{Code and Runtimes}
\subsection{C-CAPT (2D)}
\subsection{C-CAPT (3D)}
\subsection{D-CAPT}

\end{document}